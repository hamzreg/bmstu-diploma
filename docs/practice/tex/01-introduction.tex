\chapter*{ВВЕДЕНИЕ}
\addcontentsline{toc}{chapter}{ВВЕДЕНИЕ}

Данные процессов операционной системы хранятся в оперативной памяти. При ее недостатке производительность системы может снижаться. Одним из способов увеличения объема доступной оперативной памяти является сжатие данных \cite{swapping}. На качество сжатия помимо алгоритма сжатия влияет структура данных. Оценить качество сжатия, не тратя ресурсы на его выполнение, позволяет значение информационной энтропии исходных данных~\cite{theorem}.

Во время выполнения выпускной квалификационной работы была разработана оптимизация метода сжатия страниц памяти с использованием подсчета информационной энтропии. Целью данной работы является разработка программного обеспечения, демонстрирующего практическую осуществимость спроектированной оптимизации метода. Для достижения поставленной цели необходимо выполнить следующие задачи:

\begin{itemize}
	\item дать формальное описание постановки задачи оптимизации метода сжатия страниц памяти;
	\item выделить требования к программному обеспечению, 
	\item выбрать средства программной реализации;
    \item описать структуру программного обеспечения;
	\item описать конфигурацию программного обеспечения.
\end{itemize}
