\addcontentsline{toc}{chapter}{РЕФЕРАТ}
\begin{essay}{}

Объектом исследования является подсчет информационной энтропии.

Цель работы заключается в классификации методов подсчета информационной энтропии.

В рамках анализа предметной области были рассмотрены основные понятия теории информации и сжатия данных, было дано определение информационной энтропии и были представлены ее свойства.

При проведении обзора существующих методов подсчета информационной энтропии были описаны метод скользящего окна и биномиальный метод. Для их сравнения были сформулированы следующие критерии: временная сложность, необходимость вычисления факториала, возможность распараллеливания вычислений и объем требуемой дополнительной памяти.

Сравнение описанных методов по сформулированным критериям показало, что в задаче оценивания коэффициента сжатия с помощью информационной энтропии предпочтительнее использовать биномиальный метод ее подсчета.

Ключевые слова: информация, теория информации, информационная энтропия, сжатие данных, метод скользящего окна, биномиальный метод.

\end{essay}