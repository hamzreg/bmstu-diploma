\chapter{Исследовательский раздел}

Описание цели исследования. Мои гипотезы:

\begin{itemize}
    \item вычисляемая мной энтропия должна коррелировать с показателями качества сжатия;
    \item вычисляемая мной энтропия должна выполняться быстрее, чем другие методы;
    \item время вычисления энтропии должно быть значительно меньше, чем время сжатия страницы;
    \item показатели качества сжатия у бинарных и текстовых данных выше, чем у зашифрованных данных --- изображений и pdf;
    \item как время сжатия зависит от типа данных? может быть, гипотеза такая: на время сжатия в большей степени влияет выбранный алгоритм сжатия.
\end{itemize}

\section{Исследование корреляции}

Матрицы корреляций и их пояснение.

\section{Исследование времени подсчета}

Диаграммы времени.

Круговые диаграммы с процентами.

\section{Исследование зависимости характеристик от типов хранимых в памяти данных}

\subsection{Исследование показателей качества сжатия}

???

\subsection{Исследование времени сжатия}

???

\section*{Вывод}

Вычисляемая мной энтропия коррелирует с показателями качества сжатия.

Какой вывод делать по времени? Может быть, ilog2 в ядре оптимизирован (битовые операции) и очень быстрый.

Время вычисления энтропии составляет x\% от времени записи страницы на диск, время сжатия составляет y\% от времени сжатия страницы, x меньше y в z раз.

Показатели качества сжатия у бинарных и текстовых данных выше, чем у зашифрованных данных --- изображений и pdf.

Подтвердить или опровергнуть гипотезу о зависимости времени сжатия от типа данных.