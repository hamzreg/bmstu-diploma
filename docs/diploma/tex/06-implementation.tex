\chapter{Технологический раздел}

\section{Выбор средств программной реализации}

Для выполнения оптимизации сжатия страниц памяти была выбрана операционная система Linux в соответствии со сравнением систем, проведенном в разделе \ref{os}. Согласно анализу модулей сжатия, представленному в разделе \ref{linux-compression}, был выбран загружаемый модуль zram. В качестве языка программирования был выбран язык C \cite{c}, так как большая часть исходного кода ядра операционной системы Linux и всех ее модулей написана на данном языке программирования.

Для сборки программного обеспечения выбрана утилита GNU make \cite{make}, так как с помощью данной утилиты осуществляется сборка загружаемых модулей ядра.

В качестве среды разработки была выбрана Visual Studio Code \cite{vs-code} в связи с простотой и удобством использования.

\section{Детали реализации}

В связи с необходимостью использования целочисленных типов данных формула подсчета информационной энтропии из метода скользящего окна, описанного в разделе \ref{sliding-window} была преобразована.

Исходная формула имеет следующий вид:
\begin{equation}
    H(X) = -\sum_{i = 0}^{255} (p_{i} \cdot \log_{2}p_{i}), p_{i} = \frac{k_i}{N}.
\end{equation}

В соответствии со свойством логарифма можно записать:

\begin{equation}
    p_{i} \cdot \log_{2}p_{i} = \frac{k_i}{N} \cdot \log_{2}\frac{k_i}{N} =  \frac{k_i}{N} \cdot (\log_{2}k_i - \log_{2}N).
\end{equation}

Тогда:

\begin{equation}
    H(X) = \frac{k_1}{N} \cdot ( \log_{2}N - \log_{2}k_1) + ... + \frac{k_{255}}{N} \cdot (\log_{2}N - \log_{2}k_{255}).
\end{equation}

Для сохранения точности значения будет рассматриваться информационная энтропия, умноженная на размер страницы.

\begin{equation}
    H(X) \cdot N = k_1 \cdot ( \log_{2}N - \log_{2}k_1) + ... + k_{255} \cdot (\log_{2}N - \log_{2}k_{255}).
\end{equation}

В листинге \ref{lst:get_sw_entropy.c} приведена реализация метода скользящего окна для подсчета информационной энтропии. 

\includelistingpretty
    {get_sw_entropy.c}
    {C}
    {Реализация метода скользящего окна для подсчета информационной энтропии}

В листинге \ref{lst:write_page.c} представлена часть реализации записи страницы на диск в модуле блочного устройства.

\includelistingpretty
    {write_page.c}
    {C}
    {Часть реализация записи страницы на диск}

\section{Конфигурация программного обеспечения}

Для сборки, запуска и настройки разработанного программного обеспечения был написан make-файл, код которого показан в листинге \ref{lst:Makefile}.

\includelistingpretty
    {Makefile}
    {C}
    {Конфигурационный файл}

Конфигурационный файл предоставляет следующие возможности для работы с программным обеспечением:

\begin{itemize}
    \item получить исполняемый файл программного обеспечения --- загружаемый модуль, с помощью команды make;
    \item загрузить полученный модуль с помощью команды make load;
    \item проверить, что модуль загружен, с помощью команды make info;
    \item получить сообщения модуля в системном журнале с помощью команды make info;
    \item выгрузить модуль с помощью команды make unload.
\end{itemize}

Средствами конфигурационного файла можно выполнять следующие действия с диском zram:

\begin{enumerate}
    \item Добавить диск с помощью команды make add-disk, в результате выполнения которой на экран будет выведен идентификатор добавленного устройства или код ошибки.
    \item Получить список доступных алгоритмов сжатия для диска с идентификатором i с помощью команды make get-comp-algorithm id=i.
    \item Установить алгоритм сжатия a для диска с идентификатором i с помощью команды make set-comp-algorithm comp-algoritm=a id=i.
    \item Установить размер s диска с идентификатором i с помощью команды make set-disksize disksize=s id=i. Для указания единиц измерения используются следующие постфиксы: K --- размер в килобайтах, M --- размер в мегабайт, G --- размер в гигабайтах. При отсутствии постфикса устанавливается размер в байтах.
    \item Получить статистику памяти диска с идентификатором i с помощью команды make memory-stat id=i.
    \item Освободить память, выделенную для диска с идентификатором i, и сбросить его размер диска до нуля с помощью команды make reset-disk id=i.
    \item Удалить диск с идентификатором i с помощью команды make remove-disk id=i.
\end{enumerate}

\section*{Вывод}

В данном разделе был обоснован выбор языка программирования C и утилиты make в качестве средств программной реализации. Были представлены детали реализации программного решения. Кроме того были изложены команды для установки разработанного программного обеспечения и настройки устройства zram.
% Обосновать выбор средств программной реализации.
% Разработать и протестировать программное обеспечение, реализующее оптимизацию метода сжатия страниц памяти с использованием подсчета информационной энтропии.
% Описать конфигурацию программного обеспечения.
