\chapter{Конструкторский раздел}

\section{Функциональная модель оптимизации метода сжатия страниц}

Разрабатываемая оптимизация метода сжатия страниц памяти с использованием подсчета информационной энтропии состоит из следующих этапов:

\begin{enumerate}
	\item Вычисление значения информационной энтропии с помощью метода подсчета.
	\item Сравнение вычисленного значения информационной энтропии с пороговым значением.
	\item Сжатие данных страницы в случае, если полученное значение информационной энтропии меньше порогового значения.
\end{enumerate}

IDEF0-диаграмма первого уровня, формализующая основные этапы оптимизации сжатия страниц оперативной памяти, приведена на рисунке \ref{img:first-level}.
    
\includeimage
    {first-level}
    {f}
    {h}
    {1.0\textwidth}
    {IDEF0-диаграмма первого уровня}

\section{Требования к разрабатываемому программному обеспечению}

Согласно описанию этапов решения поставленной задачи разрабатываемое программное обеспечение должно:

\begin{itemize}
	\item вычислять информационную энтропию страницы оперативной памяти, которая является вектором $a = (a_1\text{ }a_2\text{ }\dotso\text{ }a_N)$ размером $N$, равным размеру страницы памяти, $0 \leq a_i \leq 255$;
	\item если вычисленное значение меньше порогового, сжимать данные входной страницы память, то есть, получать вектор $b = (b_1\text{ }b_2\text{ }\dotso\text{ }b_N)$ размером $N$, $0 \leq b_i \leq 255$;
	\item если вычисленное значение больше порогового или равно ему, данные входной страницы памяти не должны изменяться.
\end{itemize}

\section{Структура разрабатываемого программного обеспечения}

Структура загружаемого модуля ядра zram включает в себя следующие части:
\begin{itemize}
	\item модуль блочного устройства, который выполняет функции создания, настройки и удаления дисков, обработки операций записи и чтения страниц и получения статистики;
	\item модуль сжатия, который выполняет функции сжатия и восстановления данных, а также настройку этих операций.
\end{itemize}

Функция сжатия данных, предоставляемая модулем сжатия, вызывается в модуле блочного устройства во время обработки записи страницы на диск, как представлено на рисунке \ref{img:zram-structure}.

\includeimage
    {zram-structure}
    {f}
    {h}
    {0.55\textwidth}
    {Структура модуля zram}

Одним из выделенных к разрабатываемому программному обеспечению требованием является то, что сжатие страницы должно проводиться только в случае, если полученное значение информационной энтропии меньше порогового. Поэтому для выполнения поставленной задачи необходимо изменить функцию записи страницы на диск модуля блочного устройства. Построенная в соответствии с этим структура разрабатываемого программного обеспечения показана на рисунке \ref{img:structure}. Пунктирными стрелками на схеме обозначены возможные переходы.

\begin{figure}[H]
	\begin{center}
		\includegraphics[scale=0.6]{inc/img/structure.pdf}
	\end{center}
	\captionsetup{justification=centering}
	\caption{Структура разрабатываемого программного обеспечения}
	\label{img:structure}
\end{figure}

\section{Выбор типов и структур данных}

Входными и выходными данными является страница памяти, которая задается вектором размером, меньшим или равным размеру страницы в байтах, состоящим из значений от нуля до 255. Поэтому для представления страницы в методе подсчета информационной энтропии должен использоваться массив типа unsigned char размером, равным размеру страницы в байтах.

При выполнении операций с числами с плавающей точкой в режиме пользователя ядро перехватывает системное прерывание и переходит из режима вычислений с целыми числами в режим с плавающей точкой. Если использовать режим с плавающей точкой в пространстве ядра, необходимо сохранять и восстанавливать состояние регистров с плавающей точкой математического сопроцессора. Вычисления с числами с плавающей точкой в режиме ядра выполнять не рекомендуется \cite{love}. Поэтому для представления числовых величин должен использоваться целочисленный тип данных. 

\section{Описание оптимизации метода сжатия страниц}

В результате анализа исходного кода загружаемого модуля zram \cite{zram-code} и описания этапов разрабатываемой оптимизации метода сжатия страниц памяти была построена схема алгоритма записи страницы на диск, приведенная на рисунке \ref{img:write-page}. Псевдокод данного алгоритма представлен в листинге .

TODO: псевдокод.
TODO: написать про пороговое значение.

На разрабатываемый метод подсчета информационной энтропии накладываются следующие ограничения:

\begin{itemize}
	\item вычисляемое значение может представлять неточное значение информационной энтропии, но должно обладать корреляцией с показателями качества сжатия;
	\item все числовые операнды в методе должны быть целыми числами.
\end{itemize}

Разрабатываемый метод является оптимизацией метода скользящего окна. В методе скользящего окна информационная энтропия вычисляется следующим образом:

\begin{equation}
	H(X) = -\sum_{i = 0}^{255} (p_{i} \cdot \log_{2}p_{i}),
\end{equation}

\begin{figure}[H]
	\begin{center}
		\includegraphics[scale=0.7]{inc/img/write-page.pdf}
	\end{center}
	\captionsetup{justification=centering}
	\caption{Схема алгоритма записи страницы на диск}
	\label{img:write-page}
\end{figure}

\noindentгде $k_{i}$ --- число вхождений значения байта $i$ в массив байтов, представляющий входную страницу, $p_{i} = \frac{k_{i}}{N}$ --- вероятность появления байта в массиве байтов, $i = \overline{0, 255}$, $N$ --- размер страницы в байтах.

Тогда:

\begin{equation}
	H(X) = -\sum_{i = 0}^{255} (\frac{k_{i}}{N} \cdot \log_{2}\frac{k_{i}}{N}).
\end{equation}

В соответствии со свойством логарифма частного можно записать:

\begin{equation}
	H(X) = -\sum_{i = 0}^{255} (\frac{k_{i}}{N} \cdot (\log_{2}k_{i} - \log_{2}N)) = \sum_{i = 0}^{255} (\frac{k_{i}}{N} \cdot (\log_{2}N - \log_{2}k_{i})).
\end{equation}

Так как все числовые операнды являются целыми числами, при определении $p_{i}$ выполняется целочисленное деление и вычисляемое значение округляется. Для сохранения большей точности результатом разрабатываемого метода будет является значение информационной энтропии, умноженное на размер страницы:

\begin{equation}
	H'(X) = N \cdot H(X) = \sum_{i = 0}^{255} (k_{i} \cdot (\log_{2}N - \log_{2}k_{i})),
\end{equation}

Для сокращения времени вычисления логарифма числа вхождений значения байта $i$ в массив байтов по основанию два в разрабатываемом методе используется дополнительный массив размером $log_{2}N$, для которого верно следующее: 

\begin{itemize}
	\item индекс элемента $j$ является целым значением логарифма числа вхождений значения байта $i$, $j = \overline{log_{2}1, log_{2}N}$;
	\item значение элемента с индексом $j$ является наибольшим целым числом $x$ таким, что $log_{2}x = j$.
\end{itemize}

Для определения $\log_{2}k_{i}$ необходимо взять индекс первого элемента описанного массива, который больше или равен $k_i$.

Псевдокод алгоритма разрабатываемого метода подсчета информационной энтропии показан в листинге . Соответствующая ему схема алгоритма приведена на рисунке \ref{img:get-entropy}.

TODO: псевдокод.

\begin{figure}[H]
	\begin{center}
		\includegraphics[scale=0.7]{inc/img/get-entropy.pdf}
	\end{center}
	\captionsetup{justification=centering}
	\caption{Схема алгоритма подсчета информационной энтропии}
	\label{img:get-entropy}
\end{figure}

\section{Проектирование тестирования программного обеспечения}

Разрабатываемое программное обеспечение выполняется при записи данных на диск zram. Поэтому для тестирования программного решения необходимо моделировать процесс загрузки страниц памяти на диск. Проверка и отладка разрабатываемого программного обеспечения должны проводиться с помощью ручного тестирования. 

Для проведения тестирования выделены следующие классы эквивалентности:

\begin{enumerate}
	\item Несжимаемые страницы памяти;
	\item Сжимаемые страницы памяти.
\end{enumerate}

Несжимаемыми страницами памяти являются страницы с высоким значением информационной энтропии. В разделе \ref{relation} было доказано, что энтропия сжатых данных выше энтропии исходных данных. Поэтому тестовые наборы для первого класса эквивалентности должны включать сжатые данные. Примерами таких данных являются файлы форматов .jpg, .pdf, .mp3, .zip \cite{formats}.

Сжимаемыми страницами памяти являются страницы данных с выделяемой структурой, как было сказано в разделе \ref{relation}. Примерами таких данных являются файлы форматов .txt, .doc, .c, .h и бинарные файлы \cite{good-compression}, из которых должны составляться тестовые наборы для второго класса эквивалентности.

В результате каждого теста должна быть получена статистика сжатия тестового набора, включающая в себя следующую информацию:

\begin{itemize}
	\item объем исходных данных в байтах и страницах;
	\item объем сжатых данных в байтах и страницах;
	\item число несжимаемых страниц;
	\item число страниц с одинаковыми байтами.
\end{itemize}

Система для выполнения тестирования включает следующие компоненты:

\begin{itemize}
	\item персональный компьютер, который является хостом;
	\item виртуальная машина, на которой загружен модуль zram без разрабатываемой оптимизации;
	\item виртуальная машина, на которой загружен модуль zram с разрабатываемой оптимизацией.
\end{itemize}

Тестирование разрабатываемого программного обеспечения состоит из следующих этапов:

\begin{enumerate}
	\item Загрузить тестовый набор с хоста на диск zram виртуальной машины с работающим без разрабатываемой оптимизации модулем zram.
	\item Получить статистику выполнения предыдущего шага.
	\item Загрузить тестовый набор с хоста на диск zram виртуальной машины с работающим с разрабатываемой оптимизации модулем zram.
	\item Получить статистику выполнения предыдущего шага.
	\item Сравнить полученные статистики.
\end{enumerate}

Схема тестирующей системы показана на рисунке \ref{img:testing-scheme}.

\includeimage
    {testing-scheme}
    {f}
    {h}
    {0.5\textwidth}
    {Схема системы для тестирования}

\section*{Вывод}

В данном разделе были разработаны основные этапы оптимизации метода сжатия страниц памяти с использованием подсчета информационной энтропии. Их описание было приведено в виде псевдокода и схем алгоритмов. Были сформулированы требования к разрабатываемому программному решению. Взаимодействие компонентов системы было представлено в виде структуры программного обеспечения. Был обоснован выбор целочисленного типа данных и массива в качестве структуры, представляющей страницу памяти. Были выделены классы эквивалентности, описаны тестовые наборы для ручного тестирования, получаемая в результате статистика и тестирующая система.
