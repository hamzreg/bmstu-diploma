\begin{essay}{}
    В данной работе представлена разработка оптимизации метода сжатия страниц памяти с использованием подсчета информационной энтропии.

    В разделе \ref{analysis} представлено описание основных механизмов подсистемы памяти операционной системы, изучены сжатие данных и показатели его качества, рассмотрены понятие информационной энтропии, ее связь с показателями качества сжатия и методы ее подсчета, выполнено сравнение операционных систем, проведен обзор загружаемых модулей ядра Linux, предоставляющих возможность сжатия оперативной памяти. 

    В разделе \ref{design} изложены основные этапы оптимизации метода сжатия страниц памяти и метода подсчета информационной энтропии, спроектированы структура и тестирование программного обеспечения, реализующего оптимизацию метода.

    В разделе \ref{implementation} разработано и протестировано спроектированное программное обеспечение, описана его конфигурация.

    В разделе \ref{research} проведены исследование корреляции информационной энтропии и показателей качества сжатия, исследование соотношения времени сжатия и времени вычисления информационной энтропии и исследование зависимости показателей качества сжатия и времени сжатия от типов хранимых в памяти данных.

    Ключевые слова: сжатие оперативной памяти, коэффициент сжатия, информационная энтропия, метод скользящего окна, биномиальный метод.
\end{essay}