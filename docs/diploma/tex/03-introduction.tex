\chapter*{ВВЕДЕНИЕ}
\addcontentsline{toc}{chapter}{ВВЕДЕНИЕ}

Основным ресурсом в современном обществе является информация~\cite{society}. Для информационного общества характерен процесс компьютеризации, при котором задачи различных сфер жизни решаются с использованием электронно-вычислительной техники. Управление ресурсами электронно-вычислительной машины является задачей операционной системы.

На производительность операционной системы влияет объем доступной оперативной памяти. Одним из способов увеличения оперативной памяти является сжатие данных \cite{swapping-problem}. В целях оптимизации необходимо оценивать показатели его качества, что может быть реализовано с помощью информационной энтропии.

Целью данной работы является разработка оптимизации метода сжатия страниц памяти с использованием подсчета информационной энтропии.

Для достижения поставленной цели необходимо выполнить следующие задачи:

\begin{itemize}
	\item рассмотреть основные понятия управления памятью в операционной системе;
	\item провести анализ предметной области сжатия данных, 
	\item изучить свойства информационной энтропии и ее связь со сжатием данных;
	\item описать известные методы подсчета информационной энтропии;
	\item сформулировать основные этапы оптимизации метода сжатия страниц памяти с использованием подсчета информационной энтропии;
    \item разработать программное обеспечение, реализующее данную оптимизацию метода;
    \item протестировать разработанное программное обеспечение;
    \item исследовать корреляцию информационной энтропии и показателей качества сжатия;
	\item установить зависимости времени сжатия и коэффициента сжатия от различных типов хранимых в памяти данных;
    \item исследовать временную эффективность разработанной оптимизации.
\end{itemize}