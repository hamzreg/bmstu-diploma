\chapter*{ВВЕДЕНИЕ}
\addcontentsline{toc}{chapter}{ВВЕДЕНИЕ}

%Основным ресурсом в современном обществе является информация \cite{society}. В различных предметных областях появляются задачи, связанные с ее обработкой. Для их решения используются характеристики и подходы теории информации. Одной из таких характеристик является информационная энтропия. В лингвистике вычисление информационной энтропии применяется для определения показателей усилий, необходимых для перевода текста \cite{translation}, в информационной безопасности --- для оценки защищенности информационных систем \cite{security}, в медицине --- для диагностики шизофрении \cite{mind} и оценки уровня анестезии \cite{anesthesia}.

%С увеличением объема информации возрастают требуемый для ее хранения размер памяти и продолжительность передачи сведений. Для уменьшения размера данных и увеличения скорости их передачи используется сжатие данных \cite{data-compression}. В целях его оптимизации необходимо оценивать коэффициент сжатия, что может быть реализовано с помощью информационной энтропии.

%Целью данной работы является разработка оптимизации метода сжатия страниц памяти с использованием подсчета информационной энтропии.

%Для достижения поставленной цели необходимо выполнить следующие задачи:

%\begin{itemize}
%	\item рассмотреть основные понятия управления памятью в операционной системе;
%	\item провести анализ предметной области сжатия данных, 
%	\item изучить свойства информационной энтропии и ее связь со сжатием данных;
%	\item описать известные методы подсчета информационной энтропии;
%	\item разработать основные этапы оптимизации метода сжатия страниц памяти с использованием подсчета информационной энтропии;
%	\item установить зависимости времени сжатия и коэффициента сжатия от различных типов хранимых в памяти данных.
%\end{itemize}