\documentclass{bmstu}

\begin{document}

\chapter*{Оптимизация метода сжатия страниц памяти с использованием подсчета информационной энтропии}

\section*{Техническое задание}

\subsection{Аналитический раздел}

Описать основные понятия управления памятью в операционной системе. Провести анализ предметной области сжатия данных. Изучить связь коэффициента сжатия данных и информационной энтропии. Рассмотреть известные методы подсчета информационной энтропии. Сформулировать концепцию оптимизации сжатия страниц памяти операционной системы с использованием подсчета информационной энтропии. Формализовать постановку задачи в виде IDEF0-диаграммы.

\subsection{Конструкторский раздел}

Разработать основные этапы оптимизации метода сжатия страниц памяти с использованием подсчета информационной энтропии. Сформулировать и описать ключевые шаги метода подсчета информационной энтропии в виде схем алгоритмов. Обосновать выбор структур данных. Описать взаимодействие компонентов системы.

\subsection{Технологический раздел}

Обосновать выбор средств программной реализации. Разработать и протестировать программное обеспечение, реализующее оптимизацию метода сжатия страниц памяти с использованием подсчета информационной энтропии. Описать конфигурацию программного обеспечения.

\subsection{Исследовательский раздел}

Установить зависимости характеристик разработанного программного обеспечения (время сжатия и коэффициент сжатия) от различных типов хранимых в памяти данных.

\end{document}