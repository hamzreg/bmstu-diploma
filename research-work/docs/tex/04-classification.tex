\chapter{Классификация существующих методов подсчета}

При вычислении информационной энтропии методом скользящего окна и биномиальным методом операции сложения, умножения и логарифмирования применяются к целым числам и числам с плавающей запятой.

Для сравнения методов подсчета информационной энтропии были выделены следующие критерии оценки:

\begin{itemize}
	\item K1 --- временная сложность;
	\item K2 --- необходимость вычисления факториала;
	\item K3 --- возможность распараллеливания вычислений;
	\item K4 --- объем требуемой дополнительной памяти.
\end{itemize}

Результаты сравнения представлены в таблице \ref{tab:comparison}.

\begin{table}[h]
    \caption{Сравнение методов подсчета информационной энтропии}
    \begin{center}
        \begin{tabular}{|l|l|l|l|l|}
        		\hline
            \multicolumn{1}{|c}{\textbf{Метод}} & 
            \multicolumn{1}{|c|}{\textbf{K1}} &
            \multicolumn{1}{c|}{\textbf{K2}} &
            \multicolumn{1}{c|}{\textbf{K3}} & 
            \multicolumn{1}{c|}{\textbf{K4}} \\ \hline
            Скользящего окна &  $O(N + 2^n)$ & $-$ & $+$ & $2^n$ \\ \hline
            Биномиальный &  $O(N + n)$ & $+$ & $+$ & $2 \cdot (n + 1)$ \\ \hline
        \end{tabular}
    \end{center}
    \label{tab:comparison}
\end{table}

Таким образом, биномиальный метод подсчета информационной энтропии требует меньших вычислительных затрат по времени и по памяти.