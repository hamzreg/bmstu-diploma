\chapter{Классификация существующих методов подсчета}

\section{Сравнение и оценка методов}

При вычислении информационной энтропии методом скользящего окна и биномиальным методом операции сложения, умножения и логарифмирования применяются к целым числам и числам с плавающей запятой. 

Для сравнения методов подсчета информационной энтропии были выделены следующие критерии оценки:

\begin{itemize}
	\item K1 --- временная сложность;
	\item K2 --- необходимость вычисления факториала;
	\item K3 --- возможность распараллеливания вычислений;
	\item K4 --- необходимость дополнительной памяти;
	\item K5 --- рост числа операций с увеличением длины сообщения (?).
\end{itemize}

Результаты сравнения представлены в таблице \ref{tab:comparison}.

\begin{table}[h]
    \caption{Сравнение методов подсчета информационной энтропии}
    \begin{center}
        \begin{tabular}{|c|c|c|c|c|c|}
        		\hline
            \textbf{Метод} & \textbf{K1} & \textbf{K2} & \textbf{K3}
             & \textbf{K4} & \textbf{K5} \\ \hline
            Скользящего окна &  $O(N)$ & - & - & + & $O(exp(n))$ \\ \hline
            Биномиальный &  $O(N)$ & + & + & - & ? \\  \hline
        \end{tabular}
    \end{center}
    \label{tab:comparison}
\end{table}

\section{Вывод}